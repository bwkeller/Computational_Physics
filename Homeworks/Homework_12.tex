\documentclass[11pt]{article}
%\voffset -.5in \hoffset -.5in \textheight 9.9in \textwidth 6.4in
\usepackage{geometry,url}
\geometry{body={7in,9.3in}, centering}
\pagestyle{empty}
\usepackage{multicol}
\usepackage{amsmath}
\usepackage{enumitem}
\usepackage{hyperref}
\begin{document}
\begin{center}
\textbf{\Large Homework 12}\\
\textbf{Due: May 1, 2024}\\
\end{center}
\section*{Instructions}
Please write your code on \texttt{solid.memgalsim.space}, and store your results
in the \texttt{PHYS4420} directory in your home directory.  Make a new directory
in there, called \texttt{HW12}, and store each question as a separate
\texttt{question\_NUM.cpp} file.  For the questions not involving just code,
please send me a PDF of your answers.

This homework relies on PDE solvers that we worked on in class.  If you need to
reference the example code, it can be found here: \\
\url{https://github.com/bwkeller/Computational_Physics/blob/main/physics_examples/}

\section*{Questions}
\begin{enumerate}
    \item (35 points) Let's solve the Poisson equation for the gravitational
        potential:
        $$ \nabla^2 \Phi = 4\pi G \rho $$
        We will do this in 1 dimension, and use a domain of $[0,1]$ with
        boundary conditions $\Phi =0$.  Let's put in a density such that:
        $$ 4\pi G\rho =\begin{cases}
            1 & \text{if } 0.4 < x < 0.6 \\
            0 & \text{otherwise}
        \end{cases} $$ 
        Use Gauss-Seidel relaxation to solve this problem on a $N=256$ point grid, and with
        $10^5$ iterations.  What does the potential look like near $x=0.5$?  What
        does it look as you move towards $x=0$ and $x=1$?  You should be able to
        slightly modify the \texttt{poisson.cpp} example code in the
        \texttt{physics\_examples\/pdes} directory of the course github.
    \item (10 points) Take your solution from the previous problem, but now modify the
        density to have a hollow center:
        $$ 4\pi G\rho =\begin{cases}
            1 & \text{if } 0.3 < x < 0.35 \\
            1 & \text{if } 0.7 < x < 0.75 \\
            0 & \text{otherwise}
        \end{cases} $$ 
        What does the potential look like now?  Is it doing what you would
        expect in the hollow center (\textit{Hint: remember Newton's shell
        theorem})?  Why do you think it looks like that?
    \item (5 points) What topic did you enjoy most in this course?  Which topic
        do you wish we had covered or done in more detail?
\end{enumerate}
\end{document}
