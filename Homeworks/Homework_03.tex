\documentclass[11pt]{article}
%\voffset -.5in \hoffset -.5in \textheight 9.9in \textwidth 6.4in
\usepackage{geometry,url}
\geometry{body={7in,9.3in}, centering}
\pagestyle{empty}
\usepackage{multicol}
\usepackage{enumitem}
\usepackage{hyperref}
\begin{document}
\begin{center}
\textbf{\Large Homework 3}\\
\textbf{Due: February 16, 2024}\\
\end{center}
\section*{Instructions}
Please write your code on \texttt{solid.memgalsim.space}, and store your results
in the \texttt{PHYS4420} directory in your home directory.  Make a new directory in there, called \texttt{HW3}, and 
store each question as a separate \texttt{question\_NUM.cpp} file.
\section*{Questions}
\begin{enumerate}
    \item I have placed in each of your \texttt{PHYS4420} directories an example
        of the linked-list code we wrote together in class (the file is called
        \texttt{linkedlist.cpp}).  There is now another new option for the \texttt{data}
        float: -2 should delete an element.  Fill in the block of skeleton code
        so that the program will delete the \textit{last} element in the linked
        list when -2 is input.  Make sure that the memory used for the deleted
        element is freed using the \texttt{delete} operator.  Don't forget to
        check that the program still works after deleting an element.
    \item Now modify the same skeleton code so that the \textit{first} element
        of the linked list is removed when -2 is input.
    \item The \texttt{Isotope} structure contains four members: a string for the
        name of the element (e.g. Carbon-14), two integers for the number of
        protons $Z$ and number of neutrons $N$, and a float storing the isotope
        mass in atomic mass units.  Write a program that declares such a
        structure, and then creates an \texttt{Isotope} variable called
        \texttt{Li6}, and initialize this with the correct values for lithium-6
        (you can look them up online).  Try and include as many significant
        figures for the isotope mass as possible.  Your program should then
        print the contents of \texttt{Li6} to the screen.
    \item  Aneutronic fusion is a subset of possible fusion reactions that do
        not emit neutron radiation, making them attractive possible fusion power
        reactions (neutrons are difficult to shield against, and can cause stable
        isotopes of aluminum and iron to become radioactive).  One of these
        reactions transmutes a deuterium and lithium-6 nucleus into two helium-4
        nuclei.  Write a program that initializes a \texttt{Isotope} variable
        for each of these isotopes, verifies that the number of nucleons
        (protons and neutrons) are conserved for this reaction, and then use the
        difference in mass between the reactants and product to calculate how
        much energy is released by this reaction.
    \item Many isotopes of different elements are unstable.  Extend your
        \texttt{Isotope} structure to include information for about the
        half-life and daughter isotope for unstable isotopes.  Make sure to
        document your code with comments explaining your design choices.
        \textit{Hint: What would be the most sensible data type to use for the
        daughter isotope?}  Then write a short program that populates an
        \texttt{Isotope} variable with Uranium-238 and its daughter isotope,
        Thorium-234.  The Thorium-234 variable should also include its daughter
        isotope, Protactinium-234 (you don't need to go any deeper than this).

\end{enumerate}
\end{document}
