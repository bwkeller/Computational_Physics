\documentclass[11pt]{article}
%\voffset -.5in \hoffset -.5in \textheight 9.9in \textwidth 6.4in
\usepackage{geometry,url}
\geometry{body={7in,9.3in}, centering}
\pagestyle{empty}
\usepackage{multicol}
\usepackage{enumitem}
\usepackage{hyperref}
\begin{document}
\begin{center}
\textbf{\Large Homework 2}\\
\textbf{Due: February 9, 2024}\\
\end{center}
\section*{Instructions}
Please write your code on \texttt{solid.memgalsim.space}, and store your results
in the \texttt{PHYS4420} directory in your home directory.  Make a new directory in there, called \texttt{HW2}, and 
store each question as a separate \texttt{question\_NUM.cpp} file.
\section*{Questions}
\begin{enumerate}
    \item Write a program that asks for your weight in integer ounces and then
        converts your weight into pounds and ounces.  Have your program use the
        underscore character to indicate where to type the response. Use the
        \texttt{const} symbolic constant for the conversion factor between
        pounds and ounces.
    \item Write a program that prints out the sizes of the following data types:
        \begin{enumerate}
            \item \texttt{int}
            \item \texttt{long}
            \item \texttt{long long}
            \item \texttt{short}
            \item \texttt{char}
            \item \texttt{float}
            \item \texttt{double}
            \item \texttt{long double}
        \end{enumerate}
    \item Write a program that requests the user to enter the current Memphis
        population and the number of undergraduates at UofM (you can google
        these numbers or make them up, the program needs to have them read in
        from the user).  Store both these as integers.  Have the
        program then display the percent of the Memphis population .  The output
        should look like this:
        \begin{verbatim}
        Enter the Memphis Population: 500000
        Enter the number of UofM undergraduates: 22000
        4.40% of the people in Memphis are currently UofM undergraduates.
        \end{verbatim}
    \item Write a program that asks you to enter a car's gasoline consumption in
        litres per 100 km and converts this to miles per gallon.

    \item Write a program that requests and displays information as shown in the
        following example of output (you can use either \texttt{char} arrays or
        strings):
        \begin{verbatim}
        What is your first (and perhaps middle) name? Benjamin Walter
        What is your last name? Keller 
        What is your age? 35
        Name: Keller, Benjamin Walter
        Age: 35
        \end{verbatim}
\end{enumerate}
\end{document}
