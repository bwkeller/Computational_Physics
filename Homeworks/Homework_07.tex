\documentclass[11pt]{article}
%\voffset -.5in \hoffset -.5in \textheight 9.9in \textwidth 6.4in
\usepackage{geometry,url}
\geometry{body={7in,9.3in}, centering}
\pagestyle{empty}
\usepackage{multicol}
\usepackage{enumitem}
\usepackage{hyperref}
\begin{document}
\begin{center}
\textbf{\Large Homework 7}\\
\textbf{Due: March 24, 2024}\\
\end{center}
\section*{Instructions}
Please write your code on \texttt{solid.memgalsim.space}, and store your results
in the \texttt{PHYS4420} directory in your home directory.  Make a new directory
in there, called \texttt{HW7}, and store each question as a separate
\texttt{question\_NUM.cpp} file.  For the questions not involving just code,
please send me a PDF of your answers.
\section*{Questions}
\begin{enumerate}
    \item (10 points) There are three finite difference rules that use two
        points: the forward difference, which relies on $f_i$ and $f_{i+1}$, the
        central difference that uses $f_{i-1}$ and $f_{i+1}$, and the backwards
        difference that uses $f_{i-1}$ and $f_{i}$.  Write a program that
        calculates the first derivative of a function using each of these
        schemes, and then evaluate the first derivative of $f(x) = 1-x^2$ at
        $x=0$, trying a couple different step sizes.  What do you notice as the
        major difference between the result each method produces?  Why do you
        think that is?

    \item (20 points) Prove that the five-point stencil for calculating the
        first derivative is accurate to $\mathcal{O}(\Delta x^4)$.  You will
        need to use the Taylor series expansion and a bit of algebra.

    \item (10 points) An even higher-order method than Simpson's rule for
        evaluating integrals numerically is Boole's rule, which approximates an
        integral as:
        $$ \int_a^b f(x) dx = \frac{2}{45}\Delta x (7f_0 + 32f_1 + 12f_2 + 32f_3
        + 7f_4)$$
        Where $\Delta x = \frac{b-a}{4}$ and $f_0 = f(a)$. Write a program that
        evaluates the integral of some function using this method, and
        empirically determine it's convergence (hint: to do this without
        changing the limits of integration, you will need to run multiple
        ``steps'' of this scheme).

    \item (10 points)  A Gaussian distribution about 0 has the functional form
        $$ f(x) = \frac{1}{\sigma\sqrt{2\pi}}e^{-\frac{x^2}{2\sigma^2}}$$ Where
        $\sigma$ is the standard deviation of the distribution.  The integral of
        the Gaussian from $-\infty\rightarrow\infty$ is exactly 1.  Using the
        rectangle rule, trapezoid rule, and Simpson's rule, evaluate what
        fraction of the Gaussian's integral lies between $-\sigma$ and $\sigma$.
        The true value should be exactly $0.682689492137$.  How many steps are
        required from each method to get to an error of less than 1 part per
        million from the true value?
\end{enumerate}
\end{document}
