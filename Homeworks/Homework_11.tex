\documentclass[11pt]{article}
%\voffset -.5in \hoffset -.5in \textheight 9.9in \textwidth 6.4in
\usepackage{geometry,url}
\geometry{body={7in,9.3in}, centering}
\pagestyle{empty}
\usepackage{multicol}
\usepackage{enumitem}
\usepackage{hyperref}
\begin{document}
\begin{center}
\textbf{\Large Homework 11}\\
\textbf{Due: April 24, 2024}\\
\end{center}
\section*{Instructions}
Please write your code on \texttt{solid.memgalsim.space}, and store your results
in the \texttt{PHYS4420} directory in your home directory.  Make a new directory
in there, called \texttt{HW11}, and store each question as a separate
\texttt{question\_NUM.cpp} file.  For the questions not involving just code,
please send me a PDF of your answers.


\section*{Questions}
\begin{enumerate}
    \item (10 points) Write a program that generates random numbers from the
        following distribution:
        $$ P(x) \propto \exp\left(-\frac{\ln(x)^2}{2}\right) $$
        using the Metropolis algorithm.  Verify that this looks like a
        log-normal distribution by drawing a few thousand points and plotting a
        histogram of them.

    \item (20 points) Prove, using truncation analysis, that the forward
        upwinding scheme for linear advection:
        $$ a_i^{n+1} = a_i^n - \frac{u\Delta t}{\Delta x}\left(a_i^n -
        a_{i-1}^n\right)$$
        is stable (in other words, the first omitted term looks like diffusion
        rather than antidiffusion).

    \item (20 points) The linear advection equation is a very simplified
        approximation of the equations for fluid flows.  Naturally, real fluids
        do not flow at a constant velocity.  Relaxing the assumption of constant
        velocity gives us \textit{Inviscid Burgers' Equation}, where the quantity being
        advected is the velocity $u$:
        $$ \frac{\partial u}{\partial t} = -u\frac{\partial u}{\partial x} $$
        To examine the behaviour of the Inviscid Burgers' Equation, write a code
        to solve it using the forward upwinding scheme.  Test this code on an
        initial condition on a domain of $x=[0,1]$, containing the IC velocity
        field
        $$ u(x) = 0.1\sin(2\pi x) + 0.2 $$
        and with a grid size of $N=256$.  Using a timestep of $\Delta t=0.01$,
        see what happens to this system when it is evolved to $t=0.1$, $t=0.2$,
        and $t=0.5$.  What is the behaviour you are seeing, and why is it
        occurring?
\end{enumerate}
\end{document}
