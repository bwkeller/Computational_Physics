\documentclass[11pt]{article}
%\voffset -.5in \hoffset -.5in \textheight 9.9in \textwidth 6.4in
\usepackage{geometry,url}
\geometry{body={7in,9.3in}, centering}
\pagestyle{empty}
\usepackage{multicol}
\usepackage{enumitem}
\usepackage{hyperref}
\begin{document}
\begin{center}
\textbf{\Large Homework 4}\\
\textbf{Due: February 23, 2024}\\
\end{center}
\section*{Instructions}
Please write your code on \texttt{solid.memgalsim.space}, and store your results
in the \texttt{PHYS4420} directory in your home directory.  Make a new directory in there, called \texttt{HW4}, and 
store each question as a separate \texttt{question\_NUM.cpp} file.
\section*{Questions}
\begin{enumerate}
    \item (10 points) Write a program that reads in two integers $n$ and $k$ from the user and then
        calculates the Binomial coefficient of those numbers:
        $$ {n \choose k} = \frac{n!}{k!(n-k)!}$$
    \item (10 points) Write a program that asks the user to type in numbers (integers or
        real).  After each entry the user types, the program should report the
        cumulative sum of all entries to date.  The program should terminate
        when the user enters 0.
    \item (10 points) Write a simple menu-driven program.  The program should display a menu
        of four Physics course numbers, each labeled with a letter.  If the user
        responds with a letter other than one of the four valid choices, the
        program should prompt the user to enter a valid response until the user
        complies.  Then the program should print out the full course title.  A
        program run could look something like this:
        \texttt{
            Please select one of the following courses:
            a) PHYS4420\t b) PHYS2110
            c) PHYS4510\t d) PHYS2010
            > f
            Please enter one of a,b,c, or d: q
            Please enter one of a,b,c, or d: a
            PHYS4420 is "Computational Skills in Physics"
        }
    \item (20 points) Write a function that takes two square matrices of real numbers and
        performs a matrix multiplication on them.  Your function will need to
        take in three arguments: two 2-dimensional array pointers (in either
        pointer \texttt{**array} syntax or array \texttt{array[][]} syntax), and
        an integer indicating the order of the two matrices.  Your function
        should return a pointer to a new 2-dimensional array containing the
        result of the matrix multiplication.
\end{enumerate}
\end{document}
