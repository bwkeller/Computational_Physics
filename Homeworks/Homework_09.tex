\documentclass[11pt]{article}
%\voffset -.5in \hoffset -.5in \textheight 9.9in \textwidth 6.4in
\usepackage{geometry,url}
\geometry{body={7in,9.3in}, centering}
\pagestyle{empty}
\usepackage{multicol}
\usepackage{enumitem}
\usepackage{hyperref}
\begin{document}
\begin{center}
\textbf{\Large Homework 9}\\
\textbf{Due: April 6, 2024}\\
\end{center}
\section*{Instructions}
Please write your code on \texttt{solid.memgalsim.space}, and store your results
in the \texttt{PHYS4420} directory in your home directory.  Make a new directory
in there, called \texttt{HW9}, and store each question as a separate
\texttt{question\_NUM.cpp} file.  For the questions not involving just code,
please send me a PDF of your answers.


\section*{Questions}
\begin{enumerate}
    \item (10 points)  We examined 4 explicit methods for integrating ODEs:
        Euler's method, Predictor-Corrector, 2nd-order Runge-Kutta, and
        4th-order Runge-Kutta.  If we are integrating an equation that looks
        like:
        $$ u''(x) = f(x) $$
        and the cost for computing $f(x)$ is much larger than any of the other
        individual pieces of work done for a single step, what is the typical
        cost per step for each of these methods?  If you want to reduce the
        error on the result by a factor of $\sim 5$, how much longer will the
        result take to compute for each of the above methods? Calculate this
        with the assumption that the cost is just:
        $$ C = N(f)$$
        where $N(f)$ is the number of times $f(x)$ is evaluated.  
    \item (15 points) Using the Hamiltonian for the simple harmonic oscillator,
        where 
        $$q' = p; \qquad p' = -q$$
        verify that the Kick-Drift-Kick form of the Leapfrog/Verlet integrator
        is symplectic \textit{(hint: find the determinant of the update
        matrix)}.
    \item (25 points) You're an artillery officer in some forgotten war, and you
        must strike an enemy bunker 10km distant, and 100m higher than your
        current position.  You have total control of the angle your cannon fires
        at, as well as the velocity that the shell leaves the barrel.  If the
        $5$ kg cannonball experiences the normal acceleration due to gravity, as well
        an aerodynamic drag force given by:
        $$F_d = (0.08\;\mathrm{kg/m}) v^2 $$
        Use the shooting method to determine the initial velocity and angle
        required to fire the cannon and hit your target.
        
\end{enumerate}
\end{document}
