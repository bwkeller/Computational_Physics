\documentclass[11pt]{article}
%\voffset -.5in \hoffset -.5in \textheight 9.9in \textwidth 6.4in
\usepackage{geometry,url}
\geometry{body={7in,9.3in}, centering}
\pagestyle{empty}
\usepackage{multicol}
\usepackage{enumitem}
\usepackage{hyperref}
\begin{document}
\begin{center}
\textbf{\Large Homework 10}\\
\textbf{Due: April 14, 2024}\\
\end{center}
\section*{Instructions}
Please write your code on \texttt{solid.memgalsim.space}, and store your results
in the \texttt{PHYS4420} directory in your home directory.  Make a new directory
in there, called \texttt{HW10}, and store each question as a separate
\texttt{question\_NUM.cpp} file.  For the questions not involving just code,
please send me a PDF of your answers.


\section*{Questions}
\begin{enumerate}
    \item (10 points) The average value $\bar x$ generated by a large number of
        calls to \textit{rand()} will naturally be \texttt{RAND\_MAX/2}.  Write
        a code to empirically determine how the root-mean-square variance:
        $$ \sigma = \sqrt{\frac{1}{N}\sum_{i=1}^N(x-\bar x)^2} $$
        scale with the number of numbers drawn, $N$?

    \item (20 points) We know that the built-in random number generator
        from the C standard library generates random numbers drawn from a
        uniform distribution.  Draw $M=100$ random floating-point numbers from \texttt{rand()} and
        take their average.  Repeat this $N=10^4$ times.  What does the
        distribution of these \textit{averages} look like to you?  How does the
        root-mean-square variance of the sample of averages scale with $N$ and
        $M$?  (You can determine this empirically, no need for a proof)
    \item (20 points) The integral of an area enclosed in an
        $n$-dimensional ball (or hypersphere) with radius $R$ is something we all know for low
        $n$. For a 1-sphere, it is a line, with integral $2R$.  For a 2-sphere,
        it is a circle, with integral $\pi R^2$.  For a three-sphere, it is
        $\frac{4}{3}\pi R^3$.  The constant coefficient out front changes with
        the dimensionality of the hypersphere, and in general the equation looks
        like $V = C_n R^n$.  Use a Monte Carlo integration to
        estimate this coefficient $C_5$ for a a 5-sphere.
\end{enumerate}
\end{document}
