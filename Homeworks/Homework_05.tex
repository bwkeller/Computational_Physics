\documentclass[11pt]{article}
%\voffset -.5in \hoffset -.5in \textheight 9.9in \textwidth 6.4in
\usepackage{geometry,url}
\geometry{body={7in,9.3in}, centering}
\pagestyle{empty}
\usepackage{multicol}
\usepackage{enumitem}
\usepackage{hyperref}
\begin{document}
\begin{center}
\textbf{\Large Homework 5}\\
\textbf{Due: March 2, 2024}\\
\end{center}
\section*{Instructions}
Please write your code on \texttt{solid.memgalsim.space}, and store your results
in the \texttt{PHYS4420} directory in your home directory.  Make a new directory in there, called \texttt{HW5}, and 
store each question as a separate \texttt{question\_NUM.cpp} file.
\section*{Questions}
\begin{enumerate}
    \item (20 points) Write a program with the following functions:
        \begin{itemize}
            \item \texttt{fill\_array()}: this function takes as arguments the
                name of an array of double values and an array size.  It prompts
                the user to enter double values to be entered in the array.  It
                stops taking input when the array becomes full or when the user
                enters a non-numeric input.  It should return the actual number
                of entries (which may be less than the array size).
            \item \texttt{show\_array()}: this function takes the same arguments
                as the previous function.  It should print the contents of the
                array.
            \item \texttt{reverse\_array()}: this function takes the same
                arguments as the previous function.  It should reverse the order
                of the values stored in the array.
        \end{itemize}
        The program should use these functions to do the following tasks in
        order:
        \begin{enumerate}
            \item Fill in an array
            \item Show the array
            \item Reverse all elements in the array
            \item Show the array
            \item Reverse all elements \textit{except} the first and last
                elements
            \item Show the array
        \end{enumerate}
    \item (10 points) Take your linked list program from Homework 3 and re-write it to place
        the operations (add an element, print the list, and your choice of
        remove the first or last element) each within its own function.
    \item (10 points) Now take that linked list program and split it into three source files:
        a header containing definitions for structs and function prototypes, a
        source file for the main program, and a source file for the three
        functions.  Store this all in a directory called \texttt{question\_2}.
    \item (10 points) Write a Makefile for the linked list program in the
        \texttt{question\_2} directory.  Make sure your Makefile properly tracks
        all dependencies and builds object files separately.  You should also
        add a \texttt{clean} option to remove object files if the user specifies
        \texttt{make clean}
\end{enumerate}
\end{document}
