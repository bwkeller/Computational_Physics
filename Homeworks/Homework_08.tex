\documentclass[11pt]{article}
%\voffset -.5in \hoffset -.5in \textheight 9.9in \textwidth 6.4in
\usepackage{geometry,url}
\geometry{body={7in,9.3in}, centering}
\pagestyle{empty}
\usepackage{multicol}
\usepackage{enumitem}
\usepackage{hyperref}
\begin{document}
\begin{center}
\textbf{\Large Homework 8}\\
\textbf{Due: March 30, 2024}\\
\end{center}
\section*{Instructions}
Please write your code on \texttt{solid.memgalsim.space}, and store your results
in the \texttt{PHYS4420} directory in your home directory.  Make a new directory
in there, called \texttt{HW8}, and store each question as a separate
\texttt{question\_NUM.cpp} file.  For the questions not involving just code,
please send me a PDF of your answers.

This homework relies on ODE solvers that we worked on in class.  If you need to
reference the example code, it can be found here: \\
\url{https://github.com/bwkeller/Computational_Physics/blob/main/physics_examples/}

\section*{Questions}
\begin{enumerate}
    \item (10 points) The code for the bisection method example I showed in
        class assumes that the function is monotonically increasing as $x$
        increases.  Naturally, plenty of functions decrease as
        $x\rightarrow\infty$. Modify this code to be able to handle both
        functions that increase with increasing $x$ as well as ones that
        decrease.  Demonstrate that it works by finding the roots for two
        equations:
        $$ f(x) = \sqrt{x}-e^x\ln(x)$$
        and
        $$ f(x) = e^x\ln(x) + e^{0.5x}$$
    \item (30 points)  Implicit methods like Picard Iteration often work better
        than explicit ones for ``stiff'' differential equations, a vague and
        ill-defined group of problems that are best thought of as just
        ``problems often produce unstable numerical solutions''.  A simple
        example of a stiff ODE would be
        $$ x' = -18x $$
        With initial conditions $x(0) = 1$.  This has the solution
        $$ x(t) = e^{-18t} $$
        This huge power of $t$ means that it's very easy for small errors to
        balloon into either oscillations or just plain runaways.  For this
        problem: 
        \begin{enumerate}
            \item Show that Euler's method with a timestep of $\Delta t=0.1$
                oscillates about the true solution when integrating from $t=0$
                to $t=1$.
            \item Show that Picard iteration with a timestep of $\Delta t=0.1$,
                using 10 iterations per timestep, does not oscillate when
                integrating from $t=0$ to $t=1$.
            \item Determine how roughly (to within a factor of 10) how small a
                timestep $\Delta t$ is required to have the final relative error
                $|x_{numerical}(1)-x(1)|/x(1) < 0.1$ for Euler's method and for
                Picard Iteration.
        \end{enumerate}
    \item (10 points) Picard Iteration clearly is useful in some cases, like the
        stiff equation above.  However, working with a fixed number of
        iterations isn't ideal.  In some cases, we will iterate too long,
        wasting precious time.  In others, we won't iterate long enough, and get
        results that are incorrect.  Modify the Picard iteration code you used
        in the previous question to instead iterate until the solution is
        converged to a tolerance of $\epsilon=10^{-6}$ (in other words, that the last
        iteration changed the result by less than $\epsilon$).  Show that for a
        timestep of $\Delta t=0.01$, this results in far fewer than 10
        iterations per step.  Show that for a timestep of $\Delta t=0.1$, it
        results in far more iterations, but produces a better solution closer to
        the analytic one.
\end{enumerate}
\end{document}
