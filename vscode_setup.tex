\documentclass[11pt]{article}
%\voffset -.5in \hoffset -.5in \textheight 9.9in \textwidth 6.4in
\usepackage{geometry,url}
\geometry{body={7in,9.3in}, centering}
\pagestyle{empty}
\usepackage{multicol}
\usepackage{enumitem}
\usepackage{hyperref}
\begin{document}
\begin{center}
\textbf{\Large Setting up VS Code on Your Machine}\\
\end{center}
\section*{Why VS Code?}
Developing code in any language requires a set of tools: a text editor to write the code in, a compiler or interpreter 
to translate your code into machine language that the CPU can execute, and tools like debuggers and linters to help write that
code.  

Getting all these tools set up can be a pain.  It's even more of a pain if we need to get tools set up for different operating
systems (Linux, Windows, and MacOS).  Common C compilers like clang and GCC are a huge headache to get working on Windows.

To streamline all this, I've set up a Linux server at \textit{solid.memgalsim.space} with all the tools needed to build C++ programs.
This will let me see your code for homework easily too, and help you debug as well.  To connect to this server, and edit our code that
lives on this server, we'll use the popular editor developed by Microsoft, \textit{VS Code}.  This will let us connect to a remote 
server, run commands from the terminal, and edit code as well.  This document should step you through installing VS Code and connecting to 
\textit{Solid}.

\section*{TL;DR Setup}
\begin{enumerate}
    \item Install VS Code. \href{https://code.visualstudio.com/download}{https://code.visualstudio.com/download}
    \begin{itemize}
        \item Install OpenSSH Client for Windows if necessary
    \end{itemize}
    \item Install VS Code \textbf{Remote-SSH} extension: \\ \texttt{Ctrl+P} followed by \texttt{ext install ms-vscode-remote.remote-ssh} then hit enter
    \item Install VS Code \textbf{C/C++ Extension Pack} extension: \\ \texttt{Ctrl+P} followed by \texttt{ext install ms-vscode.cpptools-extension-pack} then hit enter
    \item Open Command Palette (\texttt{Ctrl+Shift+P}), use the command \texttt{Remote-SSH: Connect to Host} to SSH in to \textit{username@solid.memgalsim.space}
    \begin{itemize}
        \item If asked, select platform ``Linux''
    \end{itemize}
    \item Open Folder \texttt{Ctrl+K, Ctrl+O}, select \texttt{/home/YOUR-USERNAME/PHYS4420}
\end{enumerate}

\section*{Windows Installation}
MS Windows requires some slight extra work in order to let you connect to \textit{Solid} over SSH.  Follow the 
instructions \href{https://learn.microsoft.com/en-gb/windows-server/administration/openssh/openssh_install_firstuse}{here}
to get this installed.  You will need this working to be able to connect to \textit{Solid}.  The basic steps are:

\begin{enumerate}
    \item Open \textbf{Settings}, select \textbf{System}, then select \textbf{Optional Features}.
    \item Scan the list to see if the OpenSSH is already installed. If not, at the top of the page, select \textbf{Add a feature}, then:
    \item Find \textbf{OpenSSH Client}, then select \textbf{Install}
\end{enumerate}

\textit{You do not need to install the OpenSSH Server, just the Client!}

\section*{Connecting to \textit{Solid}}
Solid is a small Linux server that should be enough for all the code development we'll be doing in this class.  Your username
is the same as your UofM UUID (in other words, your email address).  Your initial temporary password has been emailed to you, 
and you can change that password as soon as you log in using the command \texttt{passwd}.

\end{document}
